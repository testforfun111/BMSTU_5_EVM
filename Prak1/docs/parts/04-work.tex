\chapter{Индивидуальное задание}

Вариант 20: 
Устройство формирования индексов SQL EXCEPT. Сформировать в хост-подсистеме и передать в SPE 256 записей множества A (случайные числа в диапазое 0..1024) и 256 записей множества B (случайные числа в диапазоне 0..1024). Сформировать в SPE множество C = A not B. Выполнить тестирование работы SPE, сравнив набор ключей в множестве C с ожидаемым.

\chapter{Экспериментальная часть}




\section{Код программного обеспечения}

\subsection{Host}

\begin{lstlisting}[label=lst:host,caption=Измененный код хост-системы под индивидульное задание]
#include <iostream>
#include <iterator>
#include <string>
#include <regex>
#include <sstream>
#include <fstream>
#include <ctime>
#include "host_main.h"
using namespace std;

uint64_t MAX_RECORD = 256;
uint64_t MAX = 1024;

int main(int argc, char** argv)
{
	srand(time(NULL));
	ofstream log("practicum.log"); //поток вывода сообщений
	gpc *gpc64_inst; //указатель на класс gpc
	
	//Инициализация gpc
	if (argc<2) 
	{
		log<<"Использование: host_main <путь к файлу rawbinary>"<<endl;
		return -1;
	}
	
	//Захват ядра gpc и запись sw_kernel
	gpc64_inst = new gpc();
	log<<"Открывается доступ к "<<gpc64_inst->gpc_dev_path<<endl;
	if (gpc64_inst->load_swk(argv[1])==0) {
		log<<"Программное ядро загружено из файла "<<argv[1]<<endl;
	}
	else {
		log<<"Ошибка загрузки sw_kernel файла << argv[1]"<<endl;
		return -1;
	}
	//готов данные
	uint64_t A[MAX_RECORD];
	uint64_t B[MAX_RECORD];
	uint64_t C[MAX_RECORD];
	ifstream fileA("./test/A");
	ifstream fileB("./test/B");
	ifstream fileC("./test/C");
	uint64_t x;
	int count = 0;
	while(fileA >> x)
	A[count++] = x;
	count = 0;
	while(fileB >> x)
	B[count++] = x;
	count = 0;
	while(fileC >> x)
	C[count++] = x;
	fileA.close();
	fileB.close();
	fileC.close();
	// Инициализация
	gpc64_inst->start(__event__(update)); //обработчик вставки 
	
	for (uint64_t j = 0; j < MAX_RECORD; j++) 
	{	
		gpc64_inst->mq_send(A[j]); 
	}
	for (uint64_t j = 0; j < MAX_RECORD; j++) 
	{	
		gpc64_inst->mq_send(B[j]); 
	}
	//Терминальный символ
	gpc64_inst->mq_send(-1ull);
	
	gpc64_inst->start(__event__(expect_not)); //обработчик запроса not 
	
	uint64_t k = gpc64_inst->mq_receive();
	cout<< k << endl;
	cout<< gpc64_inst->mq_receive() << endl;
	cout<< gpc64_inst->mq_receive() << endl;
	
	count = 0;
	int err = 0;
	while (1) 
	{
		uint64_t key = gpc64_inst->mq_receive();
		if (key == -1ull) break;
		if (C[count++] != key)
		err++;
	}
	cout << "Ошибок: " << err <<endl;
	return 0;
}
\end{lstlisting}

\subsection{sw\_kernel}

\begin{lstlisting}[label=lst:swkernel,caption=Измененный код sw\_kernel под индивидульное задание]
#include <stdlib.h>
#include <ctime>
#include "lnh64.hxx"
#include "gpc_io_swk.h"
#include "gpc_handlers.h"
// #include "iterators.h"
// #include "common_struct.h"
#include "compose_keys.hxx"

#define __fast_recall__
#define LEFT_STRUCT 1
#define RIGHT_STRUCT 2
#define RESULT_STRUCT 4
#define MAX 256

extern lnh lnh_core;
volatile unsigned int event_source;
uint64_t LNH_key;
int main(void) {
	/////////////////////////////////////////////////////////
	//                  Main Event Loop
	/////////////////////////////////////////////////////////
	//Leonhard driver structure should be initialised
	lnh_init();
	for (;;) {
		//Wait for event
		event_source = wait_event();
		switch(event_source) {
			/////////////////////////////////////////////
			//  Measure GPN operation frequency
			/////////////////////////////////////////////
			case __event__(update) : update(); break;
			case __event__(expect_not) : expect_not(); break;
		}
		set_gpc_state(READY);
	}
}

//-------------------------------------------------------------
//      Вставка ключа и значения в структуру
//-------------------------------------------------------------

void update()
{
	lnh_del_str_sync(LEFT_STRUCT);
	lnh_del_str_sync(RIGHT_STRUCT);
	int count = 0;
	while(1)
	{
		LNH_key = mq_receive();
		if (LNH_key==-1ull) break;
		if (count < MAX)
		lnh_ins_async(LEFT_STRUCT, LNH_key, LNH_key); //Вставка в таблицу с типизацией uint64_t
		else 
		{
			lnh_ins_async(RIGHT_STRUCT, LNH_key, LNH_key); //Вставка в таблицу с типизацией uint64_t
		}
		count++;
	} 
	
}

//-------------------------------------------------------------
//      not
//-------------------------------------------------------------

void expect_not() 
{
	
	mq_send(lnh_get_num(LEFT_STRUCT));
	mq_send(lnh_get_num(RIGHT_STRUCT));
	lnh_not_sync(LEFT_STRUCT, RIGHT_STRUCT, RESULT_STRUCT);
	mq_send(lnh_get_num(RESULT_STRUCT));
	
	uint64_t count = lnh_get_num(RESULT_STRUCT);
	lnh_get_first(RESULT_STRUCT);
	for (uint64_t i = 0; i < count; i++)
	{
		mq_send(lnh_core.result.key);
		lnh_next(RESULT_STRUCT, lnh_core.result.key);
	}
	mq_send(-1ull);
	
}

\end{lstlisting}

\section{Тестирование программного обеспечения}

Тестирование пройдено успешно.

\section{Вывод}
В ходе практикума было проведено ознакомление с типовой структурой двух взаимодействующих программ: хост-подсистемы и программного ядра sw\_kernel.
Была разработана программа для хост-подсистемы и обработчика программного ядра, выполняющия действия, описанные в индивидуальном задании